\chapter{Lessons Learned}
\label{chapter:learnt-lessons}

Challenges and problems come and go, however the knowledge acquired while working on ameliorating them stays.
The lessons we learned while porting QEMU to Unikraft become valuable assets going forward.

The first and most important takeaway is that in order to achieve a great project and enjoy working on it, you have to take some technical debt so that you can firstly have something that functions.
On top of this foundation that provided the basic functionality, you can then develop something that also looks nice.
After achieving the functionality and the niceness, you can start thinking about performance and efficiency.
To put it simple: \textit{Make it work, then make it right, then make it fast}.

The second lesson learned is that having multiple tracks at hand and being able to context switch through them can help with productivity.
There's no such thing as getting bored when you have to think about the multitude of libraries that need porting and updating in order to have \textit{lib-qemu} up and running.
