\chapter{Related Work}
\label{chapter:related-work}

\section{QEMU Across Platforms}
\label{sec:qemu-across-platforms}

QEMU's accelerator, host operating system and host architecture support is extensive: \labelindexref{Table}{table:qemu-acc-support} illustrates just that, however it does not account for the host operating system running virtualized, let that be a stripped down traditional Linux VM, or a more exotic MiniOS unikernel.

\begin{center}
\begin{table}[htb]
  \caption{QEMU support across host OS and architecture \cite{qemu-support}}
  \begin{tabular}{l*{4}{c}r}
    Accelerator & Host OS & Host Architecure \\
    \hline
    KVM & Linux & AArch64, x64, x86, others  \\
    Xen & Linux in dom0 & AArch64, AArch32, x64, x86  \\
    HVF & MacOS & x64, AArch64 \\
    WHPX & Windows & x64, x86 \\
    NVMM & NetBSD & x64, x86 \\
    TCG & Linux, Windows, MacOS, others & AArch64, AArch32, x64, x86, others
  \end{tabular}
  \label{table:qemu-acc-support}
\end{table}
\end{center}

For the purpose of our project, the last two approaches are of interest, since Xen, following the dom0 disaggregation philosophy \cite{dom0-disaggregation}, wishes to spawn an as slim as possible device model stubdomain for each HVM domU that is running at a given moment of time.
Naturally, a such QEMU instance runs unprivileged and virtualized, so the usual setup of the process completely differs from the one QEMU was originally built for.

\labelindexref{Figure}{img:disk-writes} shows a basic example of QEMU emulating a disk device doing writes in the two scenarios mentioned above, and its overhead in doing so.
One can observe that once with the implementation, the performance also massively differs, therefore we focus on the separate semantics going forward.

\fig[scale=0.7]{src/img/disk-writes.pdf}{img:disk-writes}{Disk writes speed in Linux VMs vs MiniOS unikernels \cite{linux-stubdomain}}

\subsection{MiniOS Unikernels}
\label{subsec:minios-unikernels}

Spawning a MiniOS unikernel tightly coupled with the QEMU source code was the primary approach when the idea of device model stubdomains came about in Xen, more than 10 years ago: MiniOS provided the necessary minimalistic features that the back then QEMU needed and was lightweight enough to offer great performance, even with multiple HVMs running concurrently.
However, the virtualization and emulation sphere faced great improvements over a little period of time, and soon, MiniOS was found to be unable to mirror some of the more advanced features of QEMU.
Specifically, the libc MiniOS used, Newlib \cite{newlib}, a library for embedded systems, struggled to supply demanded features, and soon, the MiniOS unikernel stubdomain became an antique relic able to run up to QEMU's 0.97 version.

\subsection{Linux VMs}
\label{subsec:linux-vms}

As a result of the MiniOS unikernels becoming obsolete, the XenProject community had to devise a new plan.
They had two options: either port a more comprehensive libc to be linked against MiniOS (but also work in the core libraries to implement new syscall shims) or run QEMU in a Linux virtual machine, having guaranteed support for the novel features found in the upstream software.
The downside of this approach is the heaviness of the VM, which even stripped down occupies way too much memory, but a trade-off had to be made, and XenProject's principal consumer, Qubes OS, adopted the idea and uses it in their ecosystem.
