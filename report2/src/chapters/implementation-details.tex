\chapter{Implementation Details}
\label{chapter:implementation-details}

Porting any piece of software as an external library to Unikraft means molding its build system to Unikraft's rather than the other way around.
Unikraft's build lifecycle consists of multiple steps:

\begin{enumerate}
    \item \textbf{Configuring} the unikernel (using Kconfig \cite{kconfig} for ease of selection and dependency check).
    \item \textbf{Fetching} the remote library code (through complex \textit{make} rules saved in a \textit{Makefile.uk} file).
    \item \textbf{Preparing} the fetched library code (as stated in \textit{Makefile.uk}).
    \item \textbf{Compiling} the external library code together with Unikraft's core libraries (using the \textit{make} variables populated by \textit{Makefile.uk}).
    \item \textbf{Linking} the final unikernel image.
  \end{enumerate}

Depending on the library, extra steps must be added: patching, creating glue code in order to ensure (binary) compatibility with Unikraft or auto-generating sources and headers.
Nevertheless, the magic happens via two very important files, \textit{Config.uk}, containing the dependencies of the library, and \textit{Makefile.uk}, consisting of the \textit{make} rules that fetch, configure and build the right sources.

\section{\textit{lib-qemu} Implementation}
\label{sec:impl-lib-qemu}

Bringing QEMU to Unikraft followed the usual build lifecycle of a C/C++ open source library, as described before.
Firstly, we laid down its dependencies and registered them to the menuconfig system, as pictured in \labelindexref{Listing}{lst:qemu-config-uk}.

\lstset{language=make,caption=\textit{lib-qemu}'s \textit{Config.uk},label=lst:qemu-config-uk}
\begin{lstlisting}
config LIBQEMU
    bool "QEMU: An emulation library"
    default n
    select LIBMUSL
    select LIBGLIB
    select LIBPCRE2
    select LIBZLIB
    select LIBPIXMAN
    select LIBXENTOOLS
\end{lstlisting}

Secondly, we registered the library, fetched and patched its code with the necessary changes needed by the QubesOS community, shown in \labelindexref{Listing}{lst:qemu-makefile-uk-1}

\lstset{language=make,caption={\textit{lib-qemu}'s registration, fetching and patching from \textit{Makefile.uk}},label=lst:qemu-makefile-uk-1}
\begin{lstlisting}
####################################################################
# Library registration
####################################################################
$(eval $(call addlib_s,libqemu,$(CONFIG_LIBQEMU)))
    
####################################################################
# Sources
####################################################################
LIBQEMU_VERSION=8.1.2
LIBQEMU_URL=https://download.qemu.org/qemu-$(LIBQEMU_VERSION).tar.xz
LIBQEMU_DIR=qemu-$(LIBQEMU_VERSION)/
LIBQEMU_PATCHDIR=$(LIBQEMU_BASE)/patches
    
$(eval $(call fetch,libqemu,$(LIBQEMU_URL)))
$(eval $(call patch,libqemu,$(LIBQEMU_PATCHDIR),$(LIBQEMU_DIR)))
\end{lstlisting}

Thirdly, we collected and auto-generated sources and files as seen in QEMU's Ninja/Meson-based original build system.
This is captured in \labelindexref{Listing}{lst:qemu-makefile-uk-2}.

\lstset{language=make,caption={\textit{lib-qemu}'s auto-generation of source and header files from \textit{Makefile.uk}},label=lst:qemu-makefile-uk-2}
\begin{lstlisting}
####################################################################
# QEMU prepare
####################################################################
# Auto-generate sources and headers
$(LIBQEMU_BUILD)/.configured: $(LIBQEMU_BUILD)/.prepared
    $(call verbose_cmd,CONFIG,libqemu: $(notdir $@), \
     $(LIBQEMU_BASE)/helpers/custom_commands.sh $(LIBQEMU) && touch $@)
\end{lstlisting}

Even though QEMU's original Ninja/Meson-based build system uses \textit{autoconf} \cite{autoconf} as a first step to generate target and host configurations based on the environment in which it is called, we had to avoid calling \textit{./configure} as it makes assumptions that fail in Unikraft's freestanding context.
As a result, most of the configuration steps that were resolved automatically by the script have to be employed manually, and one of the instances in which this happens is the aforementioned target, host and device configs, that are now part of the \textit{lib-qemu} port.

The final and most important step in achieving a working device model instance is deciding what sources will take part in the final image.
Because we are concerned with compartmentalization, we made the decision that \textit{lib-qemu} should be organized in sub-libraries, due to the sheer number of files.
Each sub-library has its own \textit{Makefile.uk} which further registers its sources and headers to the build system via \textit{Makefile.rules} as shown in \labelindexref{Listing}{lst:makefile-rules}.
What's more, \textit{lib-qemu}'s sub-libraries should also be included into the main \textit{Makefile.uk} in order to benefit from the rules defined by \textit{Makefile.rules} as pictured by \labelindexref{Listing}{lst:qemu-makefile-uk-3}.

\lstset{language=make,caption={\textit{lib-qemu}'s registration of sub-libraries from \textit{Makefile.uk}},label=lst:qemu-makefile-uk-3}
\begin{lstlisting}
# Additional macros for qemu sub-libraries
include $(LIBQEMU_BASE)/Makefile.rules

#############################################################
# QEMU code -- one external Makefile per sub-lib
#############################################################
include $(LIBQEMU_BASE)/Makefile.uk.qemu.authz
include $(LIBQEMU_BASE)/Makefile.uk.qemu.block
include $(LIBQEMU_BASE)/Makefile.uk.qemu.blockdev
include $(LIBQEMU_BASE)/Makefile.uk.qemu.chardev
include $(LIBQEMU_BASE)/Makefile.uk.qemu.common
include $(LIBQEMU_BASE)/Makefile.uk.qemu.crypto
include $(LIBQEMU_BASE)/Makefile.uk.qemu.event-loop-base
include $(LIBQEMU_BASE)/Makefile.uk.qemu.fdt
include $(LIBQEMU_BASE)/Makefile.uk.qemu.gdb_softmmu
include $(LIBQEMU_BASE)/Makefile.uk.qemu.gdb_user
include $(LIBQEMU_BASE)/Makefile.uk.qemu.hwcore
include $(LIBQEMU_BASE)/Makefile.uk.qemu.io
include $(LIBQEMU_BASE)/Makefile.uk.qemu.migration
include $(LIBQEMU_BASE)/Makefile.uk.qemu.qemuutil
include $(LIBQEMU_BASE)/Makefile.uk.qemu.qemu-x86_64-softmmu
include $(LIBQEMU_BASE)/Makefile.uk.qemu.qmp
include $(LIBQEMU_BASE)/Makefile.uk.qemu.qom
include $(LIBQEMU_BASE)/Makefile.uk.qemu.qos
\end{lstlisting}

\section{\textit{lib-glib} Implementation}
\label{sec:impl-lib-glib}

GLib \cite{glib} is a C utility open source library on which QEMU heavily relies for data types, macros and more.
Besides using the main GLib library, QEMU also requires porting GModule \cite{gmodule}, for dynamically loading modules.
Because of the general organization of the library's source code\footnote{GLib's tarballs deliver multiple additional libraries: GObject, GModule and GIO}, we brought GLib to Unikraft with the option of enabling/disabling GModule as the user sees fit.
This is achieved in \labelindexref{Listing}{lst:glib-config-uk} by transforming the library into a mini menuconfig.

\lstset{language=make,caption=\textit{lib-glib}'s \textit{Config.uk},label=lst:glib-config-uk}
\begin{lstlisting}
menuconfig LIBGLIB
    bool "GLib: a C utility library"
    default n
    depends on HAVE_LIBC
    depends on LIBPCRE2
    select LIBUKMMAP
    select LIBPOSIX_SYSINFO

if LIBGLIB

config LIBGLIB_GMODULE
    bool "Build gmodule"
    default n

endif
\end{lstlisting}

Similarly to \textit{lib-qemu}, GLib exposes an initial configuration step, that sets various macros based on the host system setup, which end up, unfortunately, falsely making assumptions about Unikraft's various capabilities.
We bypass this step and include our own headers, with the right modifications.

On the other hand, building GLib does not require all the blocks found in \textit{lib-qemu}: we simply fetch the source code, compile the collected sources (if GModule is selected, some extra files are added — see \labelindexref{Listing}{lst:glib-makefile-uk-1}) and link the library into the unikernel image.

\lstset{language=make,caption={\textit{lib-glib}'s on demand inclusion of GModule sources from \textit{Makefile.uk}},label=lst:glib-makefile-uk-1}
\begin{lstlisting}
LIBGLIB_SRCS-$(CONFIG_LIBGLIB_GMODULE) += $(LIBGLIB)/gmodule/gmodule.c
LIBGLIB_SRCS-$(CONFIG_LIBGLIB_GMODULE) += $(LIBGLIB)/gmodule/gmodule-deprecated.c
\end{lstlisting}

However, GLib is not fully freestanding: without accounting for \textit{lib-musl}, it relies on the PCRE2 library, which should also be introduced to Unikraft.

\section{\textit{lib-pcre2} Implementation}
\label{impl-lib-pcre2}

PCRE2 \cite{pcre2} is a library that offers regular expression pattern matching using Perl like syntax.
It is highly configurable and modular and presents itself as one of the easiest library to be ported to Unikraft due to the detailed documentation and design.

As it can be built for multiple encodings (UTF-8, UTF-16 or UTF-32), we went for a menuconfig approach, similar to how we managed \textit{lib-glib}.
To achieve this, we passed different preprocessing symbols matching the user's options as seen in \labelindexref{Listing}{lst:pcre2-makefile-uk-1}.

\lstset{language=make,caption={\textit{lib-pcre2}'s on demand selection of encodings from \textit{Makefile.uk}},label=lst:pcre2-makefile-uk-1}
\begin{lstlisting}
# Preprocessing symbols
LIBPCRE2_DEFINES-y += -DHAVE_CONFIG_H
LIBPCRE2_DEFINES-$(CONFIG_LIBPCRE2_CODE_UNIT_WIDTH_8) += -DPCRE2_CODE_UNIT_WIDTH=8 -DSUPPORT_PCRE2_8
LIBPCRE2_DEFINES-$(CONFIG_LIBPCRE2_CODE_UNIT_WIDTH_16) += -DPCRE2_CODE_UNIT_WIDTH=16 -DSUPPORT_PCRE2_16
LIBPCRE2_DEFINES-$(CONFIG_LIBPCRE2_CODE_UNIT_WIDTH_32) += -DPCRE2_CODE_UNIT_WIDTH=32 -DSUPPORT_PCRE2_32
LIBPCRE2_CFLAGS-y += $(LIBPCRE2_DEFINES-y)
\end{lstlisting}

\section{\textit{lib-xentools} Update}

Xen-tools \cite{xen-tools} represents a collection of libraries used by domains for interacting with the Xen hypervisor.
Some of the functionality it provides is concerned with: XenStore\footnote{XenStore is a hierarchical namespace (similar to a filesystem) shared between domains.}, event channels\footnote{Event channels are the basic primitive provided by Xen for event notifications, similar to hardware interrupts.}, grant tables\footnote{Grant tables provide a generic mechanism to memory sharing between domains.}, hypercalls\footnote{Hypercalls are a software trap from a domain to the hypervisor.}, privileged foreign mappings, device models and logging.
It is part of the Xen hypervisor source tree \cite{xen-source-tree} but acts as a separate subsystem.

Because Unikraft started as a Xen project, it has an early port of Xen-tools that dates back to 2019, the 4.13 release.
Nonetheless, that needs to be updated to a newer version in order to match \textit{lib-qemu} and the environment in which the stubdomain will be tested, a 4.18 Xen hypervisor running an Ubuntu 24.04 dom0, ergo, a bump to the 4.18 release for Xen-tools is necessary.

Updating the library meant revisiting the sources that were included, fixing paths for the ones that were moved and removing the ones that were reorganized.
An example to follow this idea would be the split of \textit{libxenctrl} into \textit{libxenctrl} and \textit{libxenguest}, that resulted in the decision of not including the latter due to no use-case at hand.
The Xen-tools libraries required by \textit{lib-qemu} are \textit{libxencall}, \textit{libxenctrl}, \textit{libxendevicemodel}, \textit{libxenevtchn}, \textit{libxenforeingmemory}, \textit{libgnttab}, \textit{libxenstore}, \textit{libxentoolcore} and \textit{libxentoollog}, all updated to the target 4.18 version through the usual Unikraft build lifecycle.
