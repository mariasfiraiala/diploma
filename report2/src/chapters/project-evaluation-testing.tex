\chapter{Project Evaluation and Testing}
\label{chapter:project-evaluation-testing}

Once the building blocks are assembled, we can start testing \textit{lib-qemu}'s IO emulation functionality.
To do so, we need a successful \textit{lib-qemu} build running a basic test doing device emulation.
With that set up we can then focus on comparatively evaluating Unikraft's QEMU port to a real-life application: the QubesOS Linux stubdomain running QEMU as a device model.

To this end, we can prove the theoretical advantages brought by unikernels when faced with a one purpose program that has to perform well.
As such, we'll run a Linux device model, together with an Unikraft one, collect metrics that include RAM and CPU usage and draw conclusions.
We are ultimately expecting to show beyond doubt that an Unikraft unikernel running QEMU is indeed the best option for device emulation because of security, slim memory footprint and fast responses.

\labelindexref{Figure}{img:project-evaluation-steps} demonstrates the structured project performance evaluation plan together with the meaning and importance of each step's.

\fig[scale=0.5]{src/img/project-evaluation-testing.png}{img:project-evaluation-steps}{Project evaluation and testing steps}
