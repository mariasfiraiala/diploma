\chapter{Background}
\label{chapter:background}

\section{Unikernels and Unikraft}
\label{sec:unikernels-unikraft}

Unikernels, also known as exokernels or library operating systems, come in various shapes and forms, centered around a philosophy.
The major two directions that emerge are concerned, on one hand, with portability, through syscall shims and binary patching, and on the other hand, with rewriting these abstractions.
Unikraft is part of the first category, having implemented a powerful syscall shim layer and an ELF loader library \cite{app-elfloader} capable of linking and running unmodified POSIX compliant binaries (and shared libraries) in the final VM.
While running unmodified applications sounds attractive, this comes with two downsides: the UDK must carry over all the abstractions of the Linux kernel and there's little support for running these unmodified binaries on platforms other than QEMU/KVM.

Unfortunately, the real platform winner when it comes to performance is a Type-1 hypervisor, such as Xen, for which there is no support in the Unikraft ELF loader library.
Xen removes the host OS layer, present in a micro VM architecture, as pictured in \labelindexref{Figure}{img:unikernel-vm-architecture} and achieves faster results in a paravirtualized context.

\fig[scale=0.5]{src/img/unikernel-vm-architecture.png}{img:unikernel-vm-architecture}{VM vs unikernel architecture \cite{vm-vs-unikernel}}

\section{Xen and the Art of Virtualization \cite{art-of-xen}}
\label{sec:xen}

Xen is a Type-1 (also known as native or baremetal) hypervisor, introduced in 2003 in an effort to provide the industry with a VMM capable of managing the notoriously hard to virtualize x86 architecture.
On the x86 architecture, it occupies ring 0, with the guest kernels being evicted to the unused ring 1, and the applications running in the already established ring 3.
It provides a special guest (or domain), named dom0 to fulfill tasks that Xen does not wish to implement, such as device drivers.
Therefore, the dom0 guest needs elevated privileges and has to be properly secured.
The other domains running under Xen are called domU guests and, depending on the architecture, are aware or not of the virtualized environment they are running in.
There are three main types of virtualization provided by Xen \cite{xen-virtualization-types}:

\begin{enumerate}
  \item PV (Paravirtualized) in which the guest OS is aware that it is running in a virtualized environment, and was partially rewritten to account for it.
  \item HVM (Hardware Virtualized Machine) in which the guest OS takes full advantage of the virtualization extensions introduced in newer processors (VT-x for Intel, and AMD-v for AMD) and runs unmodified.
  \item PVH (Enhanced PV) in which the guest OS takes advantage of both paravirtualization and HVM features.
\end{enumerate}

\section{Unikraft and Xen}
\label{sec:unikraft-xen}

While hardware virtualization extensions made virtualizing x86 easier, they provided help with only one component, the processor.
All other units still needed to be emulated, and Xen accomplished that by intergrating QEMU into the software stack with the whole purpose of emulating disk, network, motherboard, and PCI devices.

The first take was to run QEMU in dom0, as showed in  \labelindexref{Figure}{img:QEMU-in-dom0}, however that increases the attack surface of the privileged domain, and its task load.
The second take, and the more efficient one, was to run QEMU in its own domU, providing every HVM guest with a unique device model.
Pictured in \labelindexref{Figure}{img:QEMU-in-stubdomain}, this hyper-specialised guest is known as a stubdomain (or stubdom) and is currently built as a MiniOS unikernel against an out-of-date QEMU version.

\fig[scale=0.5]{src/img/QEMU-in-dom0.png}{img:QEMU-in-dom0}{QEMU running as a device model in dom0 \cite{model-device-stubdom}}
\fig[scale=0.5]{src/img/QEMU-in-stubdomain.png}{img:QEMU-in-stubdomain}{QEMU running as a device model in stubdomain \cite{model-device-stubdom}}

Having started as a Xen Project \cite{unikraft-xen}, Unikraft maintains a great relationship with the Xen community and the features that make it a great UDK, ease of apps and libraries porting, POSIX compliancy,  active members, also call for exchanging the MiniOS device models with Unikraft unikernels running an upstream version of QEMU.

% This thesis presents the \textbf{\project}.

% This is an example of a footnote \footnote{\url{www.google.com}}. You can see here a reference to \labelindexref{Section}{sub-sec:proj-objectives}.

% Here we have defined the CS abbreviation.\abbrev{CS}{Computer Science} and the UPB abbreviation.\abbrev{UPB}{University Politehnica of Bucharest}

% The main scope of this project is to qualify xLuna for use in critical systems.


% Lorem ipsum dolor sit amet, consectetur adipiscing elit. Aenean aliquam lectus vel orci malesuada accumsan. Sed lacinia egestas tortor, eget tristiqu dolor congue sit amet. Curabitur ut nisl a nisi consequat mollis sit amet quis nisl. Vestibulum hendrerit velit at odio sodales pretium. Nam quis tortor sed ante varius sodales. Etiam lacus arcu, placerat sed laoreet a, facilisis sed nunc. Nam gravida fringilla ligula, eu congue lorem feugiat eu.

% Lorem ipsum dolor sit amet, consectetur adipiscing elit. Aenean aliquam lectus vel orci malesuada accumsan. Sed lacinia egestas tortor, eget tristiqu dolor congue sit amet. Curabitur ut nisl a nisi consequat mollis sit amet quis nisl. Vestibulum hendrerit velit at odio sodales pretium. Nam quis tortor sed ante varius sodales. Etiam lacus arcu, placerat sed laoreet a, facilisis sed nunc. Nam gravida fringilla ligula, eu congue lorem feugiat eu.


% \subsection{Project Objectives}
% \label{sub-sec:proj-objectives}

% We have now included \labelindexref{Figure}{img:report-framework}.

% \fig[scale=0.5]{src/img/reporting-framework.pdf}{img:report-framework}{Reporting Framework}


% Lorem ipsum dolor sit amet, consectetur adipiscing elit. Aenean aliquam lectus vel orci malesuada accumsan. Sed lacinia egestas tortor, eget tristiqu dolor congue sit amet. Curabitur ut nisl a nisi consequat mollis sit amet quis nisl. Vestibulum hendrerit velit at odio sodales pretium. Nam quis tortor sed ante varius sodales. Etiam lacus arcu, placerat sed laoreet a, facilisis sed nunc. Nam gravida fringilla ligula, eu congue lorem feugiat eu.

% We can also have citations like \cite{iso-odf}.

% \subsection{Related Work}

% Lorem ipsum dolor sit amet, consectetur adipiscing elit. Aenean aliquam lectus vel orci malesuada accumsan. Sed lacinia egestas tortor, eget tristiqu dolor congue sit amet. Curabitur ut nisl a nisi consequat mollis sit amet quis nisl. Vestibulum hendrerit velit at odio sodales pretium. Nam quis tortor sed ante varius sodales. Etiam lacus arcu, placerat sed laoreet a, facilisis sed nunc. Nam gravida fringilla ligula, eu congue lorem feugiat eu.


% Lorem ipsum dolor sit amet, consectetur adipiscing elit. Aenean aliquam lectus vel orci malesuada accumsan. Sed lacinia egestas tortor, eget tristiqu dolor congue sit amet. Curabitur ut nisl a nisi consequat mollis sit amet quis nisl. Vestibulum hendrerit velit at odio sodales pretium. Nam quis tortor sed ante varius sodales. Etiam lacus arcu, placerat sed laoreet a, facilisis sed nunc. Nam gravida fringilla ligula, eu congue lorem feugiat eu.


% Lorem ipsum dolor sit amet, consectetur adipiscing elit. Aenean aliquam lectus vel orci malesuada accumsan. Sed lacinia egestas tortor, eget tristiqu dolor congue sit amet. Curabitur ut nisl a nisi consequat mollis sit amet quis nisl. Vestibulum hendrerit velit at odio sodales pretium. Nam quis tortor sed ante varius sodales. Etiam lacus arcu, placerat sed laoreet a, facilisis sed nunc. Nam gravida fringilla ligula, eu congue lorem feugiat eu.

% We are now discussing the \textbf{Ultimate answer to all knowledge}.
% This line is particularly important it also adds an index entry for \textit{Ultimate answer to all knowledge}.\index{Ultimate answer to all knowledge}

% \subsection{Demo listings}

% We can also include listings like the following:

% % Inline Listing example
% \lstset{language=make,caption=Application Makefile,label=lst:app-make}
% \begin{lstlisting}
% CSRCS = app.c
% SRC_DIR =..
% include $(SRC_DIR)/config/application.cfg
% \end{lstlisting}

% Listings can also be referenced. References don't have to include chapter/table/figure numbers... so we can have hyperlinks \labelref{like this}{lst:makefile-test}.

% \subsection{Tables}

% We can also have tables... like \labelindexref{Table}{table:reports}.

% \begin{center}
% \begin{table}[htb]
%   \caption{Generated reports - associated Makefile targets and scripts}
%   \begin{tabular}{l*{6}{c}r}
%     Generated report & Makefile target & Script \\
%     \hline
%     Full Test Specification & full_spec & generate_all_spec.py  \\
%     Test Report & test_report & generate_report.py  \\
%     Requirements Coverage & requirements_coverage &
%     generate_requirements_coverage.py   \\
%     API Coverage & api_coverage & generate_api_coverage.py  \\
%   \end{tabular}
%   \label{table:reports}
% \end{table}
% \end{center}